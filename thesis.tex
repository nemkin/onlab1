\documentclass[11pt,a4paper,oneside]{report}

\usepackage[table]{xcolor}


\usepackage[T1]{fontenc}
\usepackage[utf8]{inputenc}

\usepackage[magyar]{babel}

\selectlanguage{magyar}
\setlength{\parindent}{2em}
\setlength{\parskip}{0em}
\frenchspacing

\usepackage{setspace}
\singlespacing

\usepackage{amsfonts}
\usepackage{amsmath}
\usepackage{amssymb}
\usepackage{booktabs}
\usepackage{graphicx}


\usepackage{anysize}
\usepackage[unicode, hidelinks]{hyperref}
\usepackage{xcolor}
\usepackage{listings}

\usepackage[hang]{caption}
\usepackage[numbers]{natbib}
\bibliographystyle{acm}

\usepackage{xspace}
\usepackage{physics}
\usepackage{subcaption}
\usepackage{float}
\usepackage{dsfont}


\pagestyle{plain}
\marginsize{30mm}{30mm}{15mm}{15mm}


\author{Nemkin Viktória}
\title{Kvantum gráfbolyongások}

\begin{document}


\pagenumbering{gobble}
\hypersetup{pageanchor=false}
%--------------------------------------------------------------------------------------
%	The title page
%--------------------------------------------------------------------------------------
\begin{titlepage}
  \begin{center}
    \includegraphics[width=60mm,keepaspectratio]{figures/bme_logo.pdf}\\
    \vspace{0.3cm}
    \textbf{Budapesti Műszaki és Gazdaságtudományi Egyetem}\\
    \textmd{Villamosmérnöki és Informatikai Kar}\\
    \textmd{Számítástudományi és Információelméleti Tanszék}\\[5cm]

    \vspace{0.4cm}
    {\huge \bfseries Kvantum gráfbolyongások}\\[0.8cm]
    \vspace{0.5cm}
    \textsc{\Large MSc Önálló laboratórium 1.}\\[4cm]

    {
    \renewcommand{\arraystretch}{0.85}
    \begin{tabular}{cc}
      \makebox[7cm]{\emph{Készítette}} & \makebox[7cm]{\emph{Konzulens}}   \\ \noalign{\smallskip}
      \makebox[7cm]{Nemkin Viktória}   & \makebox[7cm]{dr. Friedl Katalin} \\
    \end{tabular}
    }

    \vfill
    {\large \today}
  \end{center}
\end{titlepage}
\hypersetup{pageanchor=false}

\tableofcontents\vfill
\pagenumbering{arabic}

\chapter{Bevezetés}

A 20. századi fizika hatalmas változásokat hozott. A relativitáselmélet mellett
megjelent a kvantummechanika, mely teljesen megváltoztatta a világnézetünket.
Ugyanebben az időszakban kezdődött el a számítástechnika hajnala is. Megjelentek
az első számítógépek és elkezdték megírni az első programokat, algoritmusokat.

Richard P. Feynman 1982-es cikkében fejtette ki, hogy a klasszikus számítógépekkel
sajnos csak exponenciális időben lehet kvantumjelenségeket szimulálni, ez azonban
túlságosan lassú a kísérletek elvégzéséhez. Ha viszont lenne egy kvantumjelenségek
alapján működő számítógépünk, akkor azzal hatékonyan lehetne szimulációkat végezni,
a fizikai kutatások elvégzéséhez. Így született meg a kvantumszámítógép gondolata.

Benioff, Deutsch, majd Bernstein és Vazirani munkássága nyomán megszületett a
kvantum számítási modell, a kvantum Turing-gép a 80-as évek végére. Ettől kezdve
az a kérdés foglalkoztatja a kvantumalgoritmusok kutatóit, hogy vajon vannak-e
használható kvantumalgoritmusok, illetve vannak-e olyanok amik jobbak mint a
klasszikus párjaik.

Shor 1990-es kvantumon alapuló prímfaktorizációs algoritmusa már használható
algoritmus lett és az RSA alapú kódolás feltörésével fenyeget, ami komoly
veszélyt, illetve komoly előnyt is jelent azoknak akiknek van kvantumszámítógépük.
Erre már felfigyeltek a nagyhatalmak, multinacionális cégek és szép lassan elkezdtek
a kvantumszámítógépek kifejlesztésével foglalkozni.

2019-ben a Google quantum supremacy bizonyítéka mutatott egy olyan kvantumalgoritmust
amely bár nem túl hasznos, de a mai legjobb szuperszámítógépeket is megverte a
Sicamore processzoruk teljesítménye.

Manapság egyre forróbb témává válik a kvantum és Magyarországon is egyre több
támogatás jut kvantummal kapcsolatos kutatásokra. A BME-n a Kvantuminformatikai
Nemzeti Labor tevékenykedik kvantumtitkosításon alapuló internet kifejlesztésével
és több tanszék, köztük a SZIT is bekapcsolódott a projektbe.

Ezen dolgozat a kvantumalgoritmusokon belül a kvantumbolyongásokkal foglalkozik.
A kvantumbolyongáson alapuló algoritmusokat azért érdemes kutatni, mert több
ismert kvantumalgoritmusnak az alapját képezik. A félév során megismerkedtem a
kvantuminformatika alapjaival, a gráfbolyongások klasszikus illetve kvantumos
változatával, valamint elkészítettem egy Pythonos keretrendszert melyben könnyen lehet
gráfbolyongásokkal kapcsolatos kísérleteket, szimulációkat végezni.

A dokumentum hátralévő fejezeteiben bemutatom a kvantuminformatika alapjait,
ismertetem a klasszikus gráfbolyongást, továbbá a kvantumbolyongás egy speciális
esetét, bemutatom az elkészített Pythonos keretrendszert, majd ismertetem
a félév során kapott szimulációs eredményeket, végül a továbbfejlesztési
lehetőségekről beszélek.
\chapter{Klasszikus gráfbolyongások}

Markov-láncok definíciója.

A Markov-láncokat irányított, súlyozott gráfokkal reprezentálhatjuk, egy
Markov folyamat pedig ezen a gráfon végzett bolyongás.

Nagyon sok gyakorlati probléma kapcsolódik a Markov-folyamatokhoz:
közgazdaságtan, játékelmélet, statisztikus fizika.

A klasszikus gráfbolyongási algoritmusok ezek mellett jól használhatóak
gráfelméleti problémák közelítő megoldására. Nagy méretű gráfok esetén egy akár
a csúcsszám méretében lineáris algoritmus lépésszáma is túlságosan nagy lehet.
Például az internetes keresőalgoritmusok (PageRank) is ilyenek.


\chapter{Kvantuminformatika}

Mielőtt a kvantumalgoritmusokkal elkezdhetünk foglalkozni, először meg kell
ismernünk a kvantuminformatika eszköztárát. A klasszikus számítógépek esetében
az információtárolás alapegysége a bit, melynek értéke lehet 0 vagy 1.
Kvantumszámítógépek esetében az alapegység a kvantumbit, vagy röviden qubit.
Kvantumbitek esetében a fizikai tulajdonságaiknak köszönhetően az állapottér a
komplex számokon értelmezett. A legkisebb, kvantumszámításra használható
állapottér 2 dimenziós.

\paragraph{Qubit} A 2 dimenziós Hilbert-tér ($H_2$) egy egységvektorát qubitnek
nevezzük. A tér bázisvektorai a $\ket{0}$ és az $\ket{1}$ ket vektorok.

Egy általános qubit tehát $c_0\ket{0} + c_1\ket{1}$ alakban írható fel, ahol
$c_0, c_1 \in{} \mathds{C}$ és $|c_0|^2 + |c_1|^2 = 1$.

\paragraph{Koordináta reprezentáció} A qubiteket koordinátákkal is lehet
reprezentálni, ekkor $\ket{0} = \begin{pmatrix} 1 \\ 0 \end{pmatrix}$, $\ket{1}
  = \begin{pmatrix} 0 \\ 1 \end{pmatrix}$ és $c_0\ket{0} + c_1\ket{1} =
  \begin{pmatrix} c_0 \\ c_1 \end{pmatrix}$.

\paragraph{Mérés} A kvantumbitek értéke ugyan bármely 1 hosszú vektor lehet a 2
dimenziós Hilbert-térben, azonban amikor ki szeretnénk olvasni az értéküket,
akkor Hilbert-tér valamely bázisvektorát fogjuk kapni. A $c_0\ket{0} +
  c_1\ket{1}$ qubit esetében $|c_0|^2$ valószínűséggel kapjuk a $\ket{0}$ vektort
és $|c_1|^2$ valószínűséggel kapjuk a $\ket{1}$ vektort.

\paragraph{Tenzorszorzat} Az $r \times s$ méretű $A$ és $t \times u$ méretű $B$
mátrixok $rt \times su$ méretű $A \otimes B$ tenzorszorzata a következőképpen
definiált:

\begin{center}
  $A = \begin{pmatrix}
      a_{11} & a_{12} & \dots  & a_{1s} \\
      a_{21} & a_{22} & \dots  & a_{2s} \\
      \vdots & \vdots & \ddots & \vdots \\
      a_{r1} & a_{r2} & \ddots & a_{rs}
    \end{pmatrix}
  $
  és
  $B = \begin{pmatrix}
      b_{11} & b_{12} & \dots  & b_{1u} \\
      b_{21} & b_{22} & \dots  & b_{2u} \\
      \vdots & \vdots & \ddots & \vdots \\
      b_{t1} & b_{t2} & \ddots & b_{tu}
    \end{pmatrix}
  $ esetén
\end{center}

\begin{center}
  $A \otimes B = \begin{pmatrix}
      a_{11}B & a_{12}B & \dots  & a_{1s}B \\
      a_{21}B & a_{22}B & \dots  & a_{2s}B \\
      \vdots  & \vdots  & \ddots & \vdots  \\
      a_{r1}B & a_{r2}B & \ddots & a_{rs}B
    \end{pmatrix}
  $
\end{center}


\paragraph{Kvantumregiszter}

2 dimenziós Hilbert-tér saját magával vett tenzorszorzata annyiszor ahány bites
a regiszter.

\section{Kvantumkapuk}

Az unitér mátrixok jelképezik a lehetséges fizikai operációkat a
kvantumbiteken. Egy ilyen unitér mátrix a Hadamard-mátrix, melyhez a
Hadamard-kapu tartozik.

\paragraph{Hadamard mátrix}

\begin{center}
  $H = \frac{1}{\sqrt{2}}\begin{pmatrix}
      1 & 1  \\
      1 & -1
    \end{pmatrix}$
\end{center}

\section{Hadamard-féle kvantumérme}

A Hadamard-féle kvantumérme egy olyan qubit, melynek a kiindulási állapota a $c
  = \ket{0}$ vagy $c \ket{1}$ vektor és ha feldobjuk, akkor a Hadamard-mátrixszal
szorzódik.

Klasszikus érmék esetében ha feldobjuk 1-szer akkor $50\%$ eséllyel kapunk
fejet, $50\%$ eséllyel pedig írást. Kvantumérmék esetében 1-szer feldobás után
hasonló a helyzet. Például $\ket{0}$-ból kiindulva $H\ket{0} =
  \frac{1}{\sqrt{2}}\ket{0} + \frac{1}{\sqrt{2}}\ket{1}$, tehát
$(\frac{1}{\sqrt{2}})^2 = \frac{1}{2}$ valószínűséggel kapunk majd $\ket{0}$,
illetve $\ket{1}$ vektort ha megmérjük a kvantumérmét a feldobás után.

Klasszikus érmék esetében 2-szer egymás után történő feldobás után a helyzet
hasonló: $50\%$ eséllyel kapunk fejet, $50\%$ eséllyel pedig írást.
Kvantumérmék esetében azonban érdekes jelenség figyelhető meg: $H^2\ket{0} =
  H(\frac{1}{\sqrt{2}}\ket{0} + \frac{1}{\sqrt{2}}\ket{1}) = \ket{0}$, hiszen a
Hadamard-mátrix négyzete az identitás mátrix. Tehát a 2. feldobás után $100\%$
valószínűséggel a kiindulási értéket fogjuk visszakapni.

A fenti megfigyelés azért nagyon érdekes, mert a 2. fejezetben ismertetett
klasszikus gráfbolyongásban a szomszédok közötti véletlenszerű választás
felfogható egy speciális érmével való dobásnak. 2-reguláris, 1 élsúlyú gráfok
esetében a hagyományos érmével dobunk és fej esetén az egyik, írás esetén a
másik szomszédba megyünk tovább. Megpróbálhatnánk ugyanezt kvantumérmével
megtenni, a fenti 2-szer dobásos kísérlet alapján várható, hogy nagyon más
eredményt fogunk kapni!

(k-reguláris, vagy általános, súlyozott gráfok esetére is lehet a fentieket
általánosítani.)

\chapter{\textbf{Kvantumséta}}

Kvantumbolyongások tanulmányozása esetén tehát először célszerű valamilyen
2-reguláris gráfon vett bolyongást tanulmányozni, például az egyenesen. A
következőkben erről lesz szó.

Tegyük fel, hogy a bolyongás az egyenesen az origóból indul és a a $k.$ lépés
után megáll! Ekkor a pozíció tárolására elegendő egy $2k+1$ méretű vektor,
hiszen legfeljebb $k$ messzire juthatunk mindkét irányban az origótól.

Jelöljük $P$-vel ezt a $2k+1$ méretű vektort és legyen
$P=(p_{-k},...,p_0,...,p_{k})$, ahol $p_i\in\mathds{C}$ $\forall i$-re és
$\sum\limits_{i=-k}^{k}|p_i|^2 = 1$. Azt mondjuk hogy annak a valószínűsége,
hogy a bolyongó éppen az $i.$ koordinátán van $|p_i|^2$. Ezáltal egy teljes
valószínűségi eloszlást kaptunk a bolyongó pozíciója szerint.

Mivel kezdetben a bolyongó az origóban van, ezért a hozzá tartozó pozícióvektor
legyen $P = (0,...,0,1,0,...,0)$, vagyis a $p_0=1$, a többi $0$.

Majd...

\begin{center}
  $\ket{\Psi_0} = P \otimes \begin{pmatrix} \frac{1}{\sqrt{2}} \\ \frac{i}{\sqrt{2}} \end{pmatrix}
    = \frac{1}{\sqrt{2}} \begin{pmatrix}
      p_{-k}         \\
      i\cdot{}p_{-k} \\
      \vdots         \\
      p_{0}          \\
      i\cdot{}p_{0}  \\
      \vdots         \\
      p_{k}          \\
      i\cdot{}p_{k}
    \end{pmatrix}
  $
\end{center}

A lépés két fázisból áll. Az első fázis a coin operátorral való szorzás, ami
egy $2 \times 2$-es almátrixokból álló $4k+2 \times 4k+2$ -es mátrix, aminek a
$2 \times 2$-es főátlójában Hadamard-mátrixok vannak:

\begin{center}
  \[ \hat{C} =
    \left(
    \begin{array}{cc|cc|cc|cc|cc|cc|cc}
        \frac{1}{\sqrt{2}} & \frac{1}{\sqrt{2}}  & 0      & 0      & \dots              & \dots               & \dots              & \dots               & \dots              & \dots               & \dots  & \dots  & 0                  & 0                   \\
        \frac{1}{\sqrt{2}} & -\frac{1}{\sqrt{2}} & 0      & 0      & \dots              & \dots               & \dots              & \dots               & \dots              & \dots               & \dots  & \dots  & 0                  & 0                   \\ \hline
        0                  & 0                   & \ddots &        & \ddots             &                     & \ddots             &                     & \ddots             &                     & \ddots &        & \vdots             & \vdots              \\
        0                  & 0                   &        & \ddots &                    & \ddots              &                    & \ddots              &                    & \ddots              &        & \ddots & \vdots             & \vdots              \\ \hline
        \vdots             & \vdots              & \ddots &        & \frac{1}{\sqrt{2}} & \frac{1}{\sqrt{2}}  & 0                  & 0                   & \ddots             &                     & \ddots &        & \vdots             & \vdots              \\
        \vdots             & \vdots              &        & \ddots & \frac{1}{\sqrt{2}} & -\frac{1}{\sqrt{2}} & 0                  & 0                   &                    & \ddots              &        & \ddots & \vdots             & \vdots              \\ \hline
        \vdots             & \vdots              & \ddots &        & 0                  & 0                   & \frac{1}{\sqrt{2}} & \frac{1}{\sqrt{2}}  & 0                  & 0                   & \ddots &        & \vdots             & \vdots              \\
        \vdots             & \vdots              &        & \ddots & 0                  & 0                   & \frac{1}{\sqrt{2}} & -\frac{1}{\sqrt{2}} & 0                  & 0                   &        & \ddots & \vdots             & \vdots              \\ \hline
        \vdots             & \vdots              & \ddots &        & \ddots             &                     & 0                  & 0                   & \frac{1}{\sqrt{2}} & \frac{1}{\sqrt{2}}  & \ddots &        & \vdots             & \vdots              \\
        \vdots             & \vdots              &        & \ddots &                    & \ddots              & 0                  & 0                   & \frac{1}{\sqrt{2}} & -\frac{1}{\sqrt{2}} &        & \ddots & \vdots             & \vdots              \\ \hline
        \vdots             & \vdots              & \ddots &        & \ddots             &                     & \ddots             &                     & \ddots             &                     & \ddots &        & 0                  & 0                   \\
        \vdots             & \vdots              &        & \ddots &                    & \ddots              &                    & \ddots              &                    & \ddots              &        & \ddots & 0                  & 0                   \\ \hline
        0                  & 0                   & \dots  & \dots  & \dots              & \dots               & \dots              & \dots               & \dots              & \dots               & 0      & 0      & \frac{1}{\sqrt{2}} & \frac{1}{\sqrt{2}}  \\
        0                  & 0                   & \dots  & \dots  & \dots              & \dots               & \dots              & \dots               & \dots              & \dots               & 0      & 0      & \frac{1}{\sqrt{2}} & -\frac{1}{\sqrt{2}} \\
      \end{array}
    \right)
  \]
\end{center}

A második fázisban definiáljuk a shift operátort, ami $2 \times 2$-es
almátrixokból áll, a főátló felett a bal felső sarokban van 1-es, a főátló
alatt a jobb felső sarokban:

\begin{center}
  \[ \hat{S} =
    \left(
    \begin{array}{cc|cc|cc|cc|cc|cc|cc}
        0                    & 0                    & \cellcolor{green!10} 1 & \cellcolor{green!10} 0 & 0                    & 0                    & \dots                  & \dots                  & \dots                  & \dots                  & \dots                & \dots                & 0                      & 0                      \\
        0                    & 0                    & \cellcolor{green!10} 0 & \cellcolor{green!10} 0 & 0                    & 0                    & \dots                  & \dots                  & \dots                  & \dots                  & \dots                & \dots                & 0                      & 0                      \\ \hline
        \cellcolor{red!10} 0 & \cellcolor{red!10} 0 & \ddots                 &                        & \ddots               &                      & \ddots                 &                        & \ddots                 &                        & \ddots               &                      & \vdots                 & \vdots                 \\
        \cellcolor{red!10} 0 & \cellcolor{red!10} 1 &                        & \ddots                 &                      & \ddots               &                        & \ddots                 &                        & \ddots                 &                      & \ddots               & \vdots                 & \vdots                 \\ \hline
        0                    & 0                    & \ddots                 &                        & 0                    & 0                    & \cellcolor{green!10} 1 & \cellcolor{green!10} 0 & \ddots                 &                        & \ddots               &                      & \vdots                 & \vdots                 \\
        0                    & 0                    &                        & \ddots                 & 0                    & 0                    & \cellcolor{green!10} 0 & \cellcolor{green!10} 0 &                        & \ddots                 &                      & \ddots               & \vdots                 & \vdots                 \\ \hline
        \vdots               & \vdots               & \ddots                 &                        & \cellcolor{red!10} 0 & \cellcolor{red!10} 0 & 0                      & 0                      & \cellcolor{green!10} 1 & \cellcolor{green!10} 0 & \ddots               &                      & \vdots                 & \vdots                 \\
        \vdots               & \vdots               &                        & \ddots                 & \cellcolor{red!10} 0 & \cellcolor{red!10} 1 & 0                      & 0                      & \cellcolor{green!10} 0 & \cellcolor{green!10} 0 &                      & \ddots               & \vdots                 & \vdots                 \\ \hline
        \vdots               & \vdots               & \ddots                 &                        & \ddots               &                      & \cellcolor{red!10} 0   & \cellcolor{red!10} 0   & 0                      & 0                      & \ddots               &                      & 0                      & 0                      \\
        \vdots               & \vdots               &                        & \ddots                 &                      & \ddots               & \cellcolor{red!10} 0   & \cellcolor{red!10} 1   & 0                      & 0                      &                      & \ddots               & 0                      & 0                      \\ \hline
        \vdots               & \vdots               & \ddots                 &                        & \ddots               &                      & \ddots                 &                        & \ddots                 &                        & \ddots               &                      & \cellcolor{green!10} 1 & \cellcolor{green!10} 0 \\
        \vdots               & \vdots               &                        & \ddots                 &                      & \ddots               &                        & \ddots                 &                        & \ddots                 &                      & \ddots               & \cellcolor{green!10} 0 & \cellcolor{green!10} 0 \\ \hline
        0                    & 0                    & \dots                  & \dots                  & \dots                & \dots                & \dots                  & \dots                  & 0                      & 0                      & \cellcolor{red!10} 0 & \cellcolor{red!10} 0 & 0                      & 0                      \\
        0                    & 0                    & \dots                  & \dots                  & \dots                & \dots                & \dots                  & \dots                  & 0                      & 0                      & \cellcolor{red!10} 0 & \cellcolor{red!10} 1 & 0                      & 0                      \\
      \end{array}
    \right)
  \]
\end{center}

\begin{center}
  \[ \hat{U} = \hat{S}\cdot{}\hat{C} =
    \left(
    \begin{array}{cc|cc|cc|cc|cc|cc|cc}
        0                    & 0                                      & \cellcolor{green!10} \frac{1}{\sqrt{2}} & \cellcolor{green!10} 0 & 0                    & 0                                      & \dots                                   & \dots                                  & \dots                                   & \dots                  & \dots                & \dots                                  & 0                                       & 0                      \\
        0                    & 0                                      & \cellcolor{green!10} \frac{1}{\sqrt{2}} & \cellcolor{green!10} 0 & 0                    & 0                                      & \dots                                   & \dots                                  & \dots                                   & \dots                  & \dots                & \dots                                  & 0                                       & 0                      \\ \hline
        \cellcolor{red!10} 0 & \cellcolor{red!10} \frac{1}{\sqrt{2}}  & \ddots                                  &                        & \ddots               &                                        & \ddots                                  &                                        & \ddots                                  &                        & \ddots               &                                        & \vdots                                  & \vdots                 \\
        \cellcolor{red!10} 0 & \cellcolor{red!10} -\frac{1}{\sqrt{2}} &                                         & \ddots                 &                      & \ddots                                 &                                         & \ddots                                 &                                         & \ddots                 &                      & \ddots                                 & \vdots                                  & \vdots                 \\ \hline
        0                    & 0                                      & \ddots                                  &                        & 0                    & 0                                      & \cellcolor{green!10} \frac{1}{\sqrt{2}} & \cellcolor{green!10} 0                 & \ddots                                  &                        & \ddots               &                                        & \vdots                                  & \vdots                 \\
        0                    & 0                                      &                                         & \ddots                 & 0                    & 0                                      & \cellcolor{green!10} \frac{1}{\sqrt{2}} & \cellcolor{green!10} 0                 &                                         & \ddots                 &                      & \ddots                                 & \vdots                                  & \vdots                 \\ \hline
        \vdots               & \vdots                                 & \ddots                                  &                        & \cellcolor{red!10} 0 & \cellcolor{red!10} \frac{1}{\sqrt{2}}  & 0                                       & 0                                      & \cellcolor{green!10} \frac{1}{\sqrt{2}} & \cellcolor{green!10} 0 & \ddots               &                                        & \vdots                                  & \vdots                 \\
        \vdots               & \vdots                                 &                                         & \ddots                 & \cellcolor{red!10} 0 & \cellcolor{red!10} -\frac{1}{\sqrt{2}} & 0                                       & 0                                      & \cellcolor{green!10} \frac{1}{\sqrt{2}} & \cellcolor{green!10} 0 &                      & \ddots                                 & \vdots                                  & \vdots                 \\ \hline
        \vdots               & \vdots                                 & \ddots                                  &                        & \ddots               &                                        & \cellcolor{red!10} 0                    & \cellcolor{red!10} \frac{1}{\sqrt{2}}  & 0                                       & 0                      & \ddots               &                                        & 0                                       & 0                      \\
        \vdots               & \vdots                                 &                                         & \ddots                 &                      & \ddots                                 & \cellcolor{red!10} 0                    & \cellcolor{red!10} -\frac{1}{\sqrt{2}} & 0                                       & 0                      &                      & \ddots                                 & 0                                       & 0                      \\ \hline
        \vdots               & \vdots                                 & \ddots                                  &                        & \ddots               &                                        & \ddots                                  &                                        & \ddots                                  &                        & \ddots               &                                        & \cellcolor{green!10} \frac{1}{\sqrt{2}} & \cellcolor{green!10} 0 \\
        \vdots               & \vdots                                 &                                         & \ddots                 &                      & \ddots                                 &                                         & \ddots                                 &                                         & \ddots                 &                      & \ddots                                 & \cellcolor{green!10} \frac{1}{\sqrt{2}} & \cellcolor{green!10} 0 \\ \hline
        0                    & 0                                      & \dots                                   & \dots                  & \dots                & \dots                                  & \dots                                   & \dots                                  & 0                                       & 0                      & \cellcolor{red!10} 0 & \cellcolor{red!10} \frac{1}{\sqrt{2}}  & 0                                       & 0                      \\
        0                    & 0                                      & \dots                                   & \dots                  & \dots                & \dots                                  & \dots                                   & \dots                                  & 0                                       & 0                      & \cellcolor{red!10} 0 & \cellcolor{red!10} -\frac{1}{\sqrt{2}} & 0                                       & 0                      \\
      \end{array}
    \right)
  \]
\end{center}

Az $i.$ lépés után az aktuális $\Psi$ vektor: $\ket{\Psi_i} =
  U^i\cdot{}\ket{\Psi_0}$.

\textbf{Mérés}

Annak a mérése, hogy a $i.$ koordinátán mekkora valószínűséggel van a bolyongó
a $j.$ lépésben:

Legyen $I_k$ egy $2k+1$ méretű 0 vektor, kivéve az $i.$ koordinátát, ahol 1.

Ekkor $\ket{p} = (\ket{I_k}\bra{I_k} \otimes \begin{pmatrix} 1 & 0 \\ 0 & 1
  \end{pmatrix}) \cdot{} \ket{\Psi_j}$.

Végül $\braket{p}$ a valószínűség, ami valós.

A fenti algoritmust implementáltam a félév során.
\chapter{Architektúra}

Ebben a fejezetben bemutatom az elkészült keretrendszert.

Programozási nyelvnek a Python 3-mat választottam. Ennek oka az, hogy nagyon
sok data science-el kapcsolatos modulja van, mely nagyban megkönnyíti a
különböző matematikai, algoritmuselméleti problémák feltárását, könnyen
iterálhatunk a különböző prototípusokon. Emellett a szintaxisa rövid, tömör,
lényegretörő programkódok megírását teszi lehetővé.

A forráskód három nagy részre bomlik:
\begin{itemize}
  \item Gráfmodellek
  \item Szimulátorok
  \item Futtatás, konfiguráció, eredmények ábragenerátora
\end{itemize}

\section{Gráfmodellek}

A félév során sokféle gráfon futtattam szimulációs kísérleteket, melyek során
több problémába ütköztem. Kezdetben úgy oldottam meg a szimulációkat, hogy a
cél gráfok szomszédossági mátrixait generáltam le, egyben a memóriában tartva
azokat és a lépések során a megfelelő csúcshoz tartozó sorokat lekérdezve.

Ezzel a módszerrel több probléma is jelentkezett. Az első gondot az okozta,
hogy a szomszédossági mátrix mérete a csúcsszám négyzetével arányos, ezért pár
ezer csúcsú gráfot már nem tudtam a memóriában tartva szimulálni. A második
probléma pedig az volt, hogy a szomszédossági mátrixos ábrázolás nagyon távol
esett az emberi szempontból természetes ábrázolástól. A kvantum bolyongásos
szimulációkat tipikusan nem véletlenszerű gráfokon szokták kipróbálni, hanem
jól ismert struktúrával rendelkező gráfokon. Ilyen gráfok például a súlyzó vagy
a ragasztott bináris fa gráfok.

A súlyzó gráf két egyforma méretű kört tartalmaz, mindkét körből kiválasztva
k-k darab csúcsot, melyek teljes páros gráfot alkotnak (a súlyzó középső
rúdját). A kör gráfokban pedig nem csak az egymás melletti csúcsok között fut
él, hanem futhat él minden i. csúcs között is. A ragasztott bináris fa gráf
pedig olyan, hogy két teljes bináris gráf leveleit szembefordítjuk és a két
oldali levelek közé egy teljes páros gráfot készítünk.

A fenti leírásból látható, hogy az ember számára természetes leírás a gráfokat
ismert részgráfok kompozitjaként adja meg. A félév során olyan architektúrát
alakítottam ki a szimulációkhoz, mely ezt a szemléletet támogatja. A
szomszédossági mátrixos tárolási mód helyett pedig a szomszédossági orákulum
megközelítést használva nagyban csökkent a memóriaigénye az alkalmazásnak.
Ennek a megközelítésnek a lényege, hogy az ismert struktúrájú gráfokra nem
tárolok a memóriában szomszédossági információt, helyette biztosítok egy
függvényt, amely a bemeneti paraméterként kapott csúcsindexre kiszámolja a vele
szomszédos csúcsok indexeit.

A félév során a következő nevesített részgráfok szomszédossági orákulumját
implementáltam:

\begin{itemize}
  \item BinaryTree
  \item Bipartite
  \item Circle
  \item Path
  \item Random
\end{itemize}

\begin{center}
  \includegraphics[width=\linewidth]{./figures/subgraph.png}
\end{center}

Ezen részgráfokból épülnek fel az alábbi kompozit gráfok:
\begin{itemize}
  \item Dumbbell
  \item GluedBinary
\end{itemize}

\begin{center}
  \includegraphics[width=0.4\linewidth]{./figures/graph.png}
\end{center}

\section{Szimulátorok}

A szimulátor osztályok közül a klasszikus tetszőleges kompozit gráfot tud
fogadni, a kvantum szimulátor jelenleg a kvantum bolyongás egy speciális
esetét, az egyenesen való bolyongást képes kezelni, mely a 2-regularitása miatt
egyszerűbben implementálható. Hosszú távú cél a k-reguláris, illetve az
általános gráfokra kiterjeszteni ezt a szimulátort.

\begin{center}
  \includegraphics[width=0.8\linewidth]{./figures/simulator.png}
\end{center}

\section{Futtatás, konfiguráció, eredmények ábragenerátora}

A fenti osztályok segítségével egy olyan keretrendszert alakítottam ki, melyben
nagyon gyorsan fel lehet 1-1 futtatást konfigurálni. A futtatás eredményeit egy
összesített Latex dokumentumba gyűjti a program. Ez tartalmazza a beadott gráf
részgráfjainak nevesített típusát, szomszédossági mátrixait, illetve a teljes
gráf szomszédossági mátrixát, valamint a szimulációk eloszlási eredményeit. A
következő fejezetben több ilyen ábrát is bemutatok.

\chapter{Szimulációk, eredmények}

Ábrák értelmezése:

A lenti ábrákon a matplotlibes plazma színezésnek megfelelően a hidegebb (kék) árnyalatok kisebb
értéket, a melegebb (lila, majd sárga) árnyalatok magasabb értékeket jelölnek. A fehér
szín a 0 értéket jelöli.

A gráfok és részgráfok ábrái n*n-es szomszédossági mátrixokat ábrázolnak. A fehér cellák a nem-élek,
a kék cellák az 1-es súlyú élek.

A szimulációk ábráinak az x tengelyén a gráfok csúcsai helyezkednek el kiterítve, index szerinti
növekvő sorrendben, az y tengelyen a szimuláció lépései helyezkednek el, alul a 0. lépés, felül
az utolsó lépés.

\section{Súlyzó gráf}

Például egy súlyzó gráfról a következő képek készültek:

\begin{figure}[H]
  \centering
  \begin{subfigure}{.3\linewidth}
    \centering
    \includegraphics[width=\linewidth]{./figures/sulyzo/subgraph_00.jpg}
    \caption{Bal kör}
    \label{fig:sub1}
  \end{subfigure}
  \begin{subfigure}{.3\linewidth}
    \centering
    \includegraphics[width=\linewidth]{./figures/sulyzo/subgraph_02.jpg}
    \caption{Középső teljes páros gráf}
    \label{fig:sub2}
  \end{subfigure}
  \begin{subfigure}{.3\linewidth}
    \centering
    \includegraphics[width=\linewidth]{./figures/sulyzo/subgraph_01.jpg}
    \caption{Jobb kör}
    \label{fig:sub3}
  \end{subfigure}
  \caption{Súlyzó gráf részgráfjai}
  \label{fig:all}
\end{figure}

\begin{figure}[H]
  \centering
  \includegraphics[width=0.5\linewidth]{./figures/sulyzo/graph.jpg}
  \caption{Súlyzó gráf}
\end{figure}

\begin{figure}[H]
  \centering
  \begin{subfigure}{.45\linewidth}
    \centering
    \includegraphics[width=\linewidth]{./figures/sulyzo/sim00.jpg}
    \caption{1 bolyongó}
  \end{subfigure}
  \begin{subfigure}{.45\linewidth}
    \centering
    \includegraphics[width=\linewidth]{./figures/sulyzo/sim01.jpg}
    \caption{10 bolyongó}
  \end{subfigure}
\end{figure}

\begin{figure}[H]
  \centering
  \begin{subfigure}{.45\linewidth}
    \centering
    \includegraphics[width=\linewidth]{./figures/sulyzo/sim02.jpg}
    \caption{100 bolyongó}
  \end{subfigure}
  \begin{subfigure}{.45\linewidth}
    \centering
    \includegraphics[width=\linewidth]{./figures/sulyzo/sim03.jpg}
    \caption{1000 bolyongó}
  \end{subfigure}
\end{figure}

\section{Ragasztott bináris gráf}

\begin{figure}[H]
  \centering
  \begin{subfigure}{.3\linewidth}
    \centering
    \includegraphics[width=\linewidth]{./figures/ragasztott_binaris/subgraph_00.jpg}
    \caption{Bal bináris fa}
  \end{subfigure}
  \begin{subfigure}{.3\linewidth}
    \centering
    \includegraphics[width=\linewidth]{./figures/ragasztott_binaris/subgraph_02.jpg}
    \caption{Középső teljes páros gráf}
  \end{subfigure}
  \begin{subfigure}{.3\linewidth}
    \centering
    \includegraphics[width=\linewidth]{./figures/ragasztott_binaris/subgraph_01.jpg}
    \caption{Jobb bináris fa}
  \end{subfigure}
  \caption{Ragasztott bináris gráf részgráfjai}
\end{figure}

\begin{figure}[H]
  \centering
  \includegraphics[width=0.5\linewidth]{./figures/ragasztott_binaris/graph.jpg}
  \caption{Ragasztott bináris gráf}
\end{figure}

\begin{figure}[H]
  \centering
  \begin{subfigure}{.45\linewidth}
    \centering
    \includegraphics[width=\linewidth]{./figures/ragasztott_binaris/sim00.jpg}
    \caption{1 bolyongó}
  \end{subfigure}
  \begin{subfigure}{.45\linewidth}
    \centering
    \includegraphics[width=\linewidth]{./figures/ragasztott_binaris/sim01.jpg}
    \caption{10 bolyongó}
  \end{subfigure}
\end{figure}

\begin{figure}[H]
  \centering
  \begin{subfigure}{.45\linewidth}
    \centering
    \includegraphics[width=\linewidth]{./figures/ragasztott_binaris/sim02.jpg}
    \caption{100 bolyongó}
  \end{subfigure}
  \begin{subfigure}{.45\linewidth}
    \centering
    \includegraphics[width=\linewidth]{./figures/ragasztott_binaris/sim03.jpg}
    \caption{1000 bolyongó}
  \end{subfigure}
\end{figure}

\section{Bolyongás az egyenesen, klasszikus és kvantum esetben}

\begin{figure}[H]
  \centering
  \begin{subfigure}{.45\linewidth}
    \centering
    \includegraphics[width=\linewidth]{./figures/quantum/classical_simulation_short.jpg}
    \caption{Klasszikus}
  \end{subfigure}
  \begin{subfigure}{.45\linewidth}
    \centering
    \includegraphics[width=\linewidth]{./figures/quantum/quantum_simulation_short.jpg}
    \caption{Kvantum}
  \end{subfigure}
\end{figure}

Kijött a szokásos klasszikus és kvantum közötti különbség, a klasszikus az
origó körül csúcsosodik, a kvantum viszont "2 púpú teve" szerű eloszlást mutat.

\begin{figure}[H]
  \centering
  \includegraphics[width=0.5\linewidth]{./figures/quantum/teve.png}
  \caption{Eloszlások különbsége}
\end{figure}

\section{Bolyongás a szakaszon, klasszikus és kvantum esetben}

A szakasz végpontjai visszatükrözik a bolyongót.

\begin{figure}[H]
  \centering
  \begin{subfigure}{.45\linewidth}
    \centering
    \includegraphics[width=\linewidth]{./figures/quantum/classical_simulation_long.jpg}
    \caption{Klasszikus}
  \end{subfigure}
  \begin{subfigure}{.45\linewidth}
    \centering
    \includegraphics[width=\linewidth]{./figures/quantum/quantum_simulation_long.jpg}
    \caption{Kvantum}
  \end{subfigure}
\end{figure}
\chapter{Jövőbeli tervek}

\begin{itemize}
  \item Elolvasni a Springer sorozatos kvantum bolyongós könyveket.
  \item Implementálni k-reguláris gráfra a bolyongást.
  \item Implementálni általános gráfra a bolyongást.
  \item Számolni hitting time-ot és mixing rate-t.
  \item Számolni a szomszédossági mátrixból sajátértékeket és azokat elemezni.
  \item Számolni határeloszlást a Markov-láncokhoz.
\end{itemize}

\nocite{*}


Források (rendesen hivatkozva majd...):
\begin{itemize}
  \item Hirvensalo könyv
  \item http://susan-stepney.blogspot.com/2014/02/mathjax.html (Ezt nyáron lecserélni rendes cikkre.)
\end{itemize}

\addcontentsline{toc}{chapter}{\bibname}
\bibliography{bib/mybib}

\end{document}