\chapter{Kvantuminformatika}

Mielőtt a kvantumalgoritmusokkal elkezdhetünk foglalkozni, először meg kell
ismernünk a kvantuminformatika eszköztárát. A klasszikus számítógépek esetében
az információtárolás alapegysége a bit, melynek értéke lehet 0 vagy 1.
Kvantumszámítógépek esetében az alapegység a kvantumbit, vagy röviden qubit.
Kvantumbitek esetében a fizikai tulajdonságaiknak köszönhetően az állapottér a
komplex számokon értelmezett. A legkisebb, kvantumszámításra használható
állapottér 2 dimenziós.

\paragraph{Qubit} A 2 dimenziós Hilbert-tér ($H_2$) egy egységvektorát qubitnek
nevezzük. A tér bázisvektorai a $\ket{0}$ és az $\ket{1}$ ket vektorok.

Egy általános qubit tehát $c_0\ket{0} + c_1\ket{1}$ alakban írható fel, ahol
$c_0, c_1 \in{} \mathds{C}$ és $|c_0|^2 + |c_1|^2 = 1$.

\paragraph{Koordináta reprezentáció} A qubiteket koordinátákkal is lehet
reprezentálni, ekkor $\ket{0} = \begin{pmatrix} 1 \\ 0 \end{pmatrix}$, $\ket{1}
  = \begin{pmatrix} 0 \\ 1 \end{pmatrix}$ és $c_0\ket{0} + c_1\ket{1} =
  \begin{pmatrix} c_0 \\ c_1 \end{pmatrix}$.

\paragraph{Mérés} A kvantumbitek értéke ugyan bármely 1 hosszú vektor lehet a 2
dimenziós Hilbert-térben, azonban amikor ki szeretnénk olvasni az értéküket,
akkor Hilbert-tér valamely bázisvektorát fogjuk kapni. A $c_0\ket{0} +
  c_1\ket{1}$ qubit esetében $|c_0|^2$ valószínűséggel kapjuk a $\ket{0}$ vektort
és $|c_1|^2$ valószínűséggel kapjuk a $\ket{1}$ vektort.

\paragraph{Tenzorszorzat} Az $r \times s$ méretű $A$ és $t \times u$ méretű $B$
mátrixok $rt \times su$ méretű $A \otimes B$ tenzorszorzata a következőképpen
definiált:

\begin{center}
  $A = \begin{pmatrix}
      a_{11} & a_{12} & \dots  & a_{1s} \\
      a_{21} & a_{22} & \dots  & a_{2s} \\
      \vdots & \vdots & \ddots & \vdots \\
      a_{r1} & a_{r2} & \ddots & a_{rs}
    \end{pmatrix}
  $
  és
  $B = \begin{pmatrix}
      b_{11} & b_{12} & \dots  & b_{1u} \\
      b_{21} & b_{22} & \dots  & b_{2u} \\
      \vdots & \vdots & \ddots & \vdots \\
      b_{t1} & b_{t2} & \ddots & b_{tu}
    \end{pmatrix}
  $ esetén
\end{center}

\begin{center}
  $A \otimes B = \begin{pmatrix}
      a_{11}B & a_{12}B & \dots  & a_{1s}B \\
      a_{21}B & a_{22}B & \dots  & a_{2s}B \\
      \vdots  & \vdots  & \ddots & \vdots  \\
      a_{r1}B & a_{r2}B & \ddots & a_{rs}B
    \end{pmatrix}
  $
\end{center}


\paragraph{Kvantumregiszter}

Az n qubites kvantumregiszter n darab qubit ($H_2$) tenzorszorzata.
$H_2^n = H_2 \otimes H_2 \otimes ... \otimes H_2$.

\section{Kvantumkapuk}

Az unitér mátrixok jelképezik a lehetséges fizikai operációkat a
kvantumbiteken. Egy ilyen unitér mátrix a Hadamard-mátrix, melyhez a
Hadamard-kapu tartozik.

\paragraph{Hadamard mátrix}

\begin{center}
  $H = \frac{1}{\sqrt{2}}\begin{pmatrix}
      1 & 1  \\
      1 & -1
    \end{pmatrix}$
\end{center}

\section{Hadamard-féle kvantumérme}

A Hadamard-féle kvantumérme egy olyan qubit, melynek a kiindulási állapota a $c
  = \ket{0}$ vagy $c = \ket{1}$ vektor és ha feldobjuk, akkor a
Hadamard-mátrixszal szorzódik.

Klasszikus érmék esetében ha feldobjuk 1-szer akkor $50\%$ eséllyel kapunk
fejet, $50\%$ eséllyel pedig írást. Kvantumérmék esetében 1-szer feldobás után
hasonló a helyzet. Például $\ket{0}$-ból kiindulva $H\ket{0} =
  \frac{1}{\sqrt{2}}\ket{0} + \frac{1}{\sqrt{2}}\ket{1}$, tehát
$(\frac{1}{\sqrt{2}})^2 = \frac{1}{2}$ valószínűséggel kapunk majd $\ket{0}$,
illetve $\ket{1}$ vektort ha megmérjük a kvantumérmét a feldobás után.

Klasszikus érmékre a második feldobás után a helyzet hasonló: $50\%$ eséllyel
kapunk fejet, $50\%$ eséllyel pedig írást. Kvantumérmék esetében azonban
érdekes jelenség figyelhető meg: $H^2\ket{0} = H(\frac{1}{\sqrt{2}}\ket{0} +
  \frac{1}{\sqrt{2}}\ket{1}) = \ket{0}$, hiszen a Hadamard-mátrix négyzete az
identitás mátrix. Tehát a 2. feldobás után $100\%$ valószínűséggel a kiindulási
értéket fogjuk visszakapni.

Klasszikus gráfbolyongások vizsgálatánál ha 2-reguláris gráfról van szó, akkor
a szomszéd véletlenszerű kiválasztása megfeleltethető egy klasszikus
érmefeldobás eredményének. Ebből a megfigyelésből kiindulva definiálhatjuk a
kvantum gráfbolyongást 2-reguláris gráfok esetén a kvantumérme
felhasználásával. A fenti eredményekből látszik, hogy nagyon más viselkedésre
lehet számítani, ezért érdemes ezzel a továbbiakban részletesen foglalkozni.

\chapter{Kvantumséta}

A kvantumbolyongásokat 2-reguláris gráfok esetén tehát a Hadamard-mátrix
segítségével definiáljuk. A következőkben az egyenesen, mint 2-reguláris gráfon
vett bolyongást fogjuk részletesen megvizsgálni.

Tegyük fel, hogy a bolyongás az egyenesen az origóból indul és a $k.$ lépés
után megáll. Ekkor a pozíció tárolására elegendő egy $(2k+1)$ méretű
állapotvektor, hiszen legfeljebb $k$ messzire juthatunk mindkét irányban a
kezdőponttól. Jelöljük ezt $\ket{P}$-vel és legyen $\ket{P}=(p_{-k}, ..., p_0,
  ..., p_{k})$, ahol $p_i\in\mathds{C}$ $\forall i$-re és $\sum\limits_{i=-k}^{k}
  |p_i|^2 = 1$. Azt mondjuk, hogy annak a valószínűsége, hogy a bolyongó éppen az
$i.$ koordinátán van $|p_i|^2$. Ezáltal egy teljes valószínűségi eloszlást
kaptunk a bolyongó pozíciója szerint.

Mivel kezdetben a bolyongó az origóban van, ezért a hozzá tartozó állapotvektor
$\ket{P_0} = (0, ..., 0, 1, 0, ..., 0)$, vagyis $p_0=1$, a többi $0$. A kvantumérme
kiindulási állapota pedig legyen $\ket{C_0} = \begin{pmatrix} \frac{1}{\sqrt{2}} \\
    \frac{i}{\sqrt{2}}\end{pmatrix}$. Ekkor a teljes rendszer állapotát e kettő
tenzorszorzata fogja megadni:

\begin{center}
  $\ket{\Psi_0} = \ket{P_0} \otimes \ket{C_0}
    = \frac{1}{\sqrt{2}} \begin{pmatrix}
      p_{-k}         \\
      i\cdot{}p_{-k} \\
      \vdots         \\
      p_{0}          \\
      i\cdot{}p_{0}  \\
      \vdots         \\
      p_{k}          \\
      i\cdot{}p_{k}
    \end{pmatrix}
  $
\end{center}

A lépés ezek után két fázisból áll. Az első fázisban az érmével dobunk. Ahhoz, hogy
a $\ket{\Psi}$ vektorban az érme állapotát frissítsük, azt az $I_{2k+1} \otimes H$
mátrixszal kell beszorozni. Ez egy olyan $(4k+2) \times (4k+2)$ -es mátrix, mely
$2 \times 2$-es almátrixokból áll és az átlójában Hadamard-mátrixok vannak:

\begin{center}
  \[ \hat{C} =
    \left(
    \begin{array}{cc|cc|cc|cc|cc|cc|cc}
        \frac{1}{\sqrt{2}} & \frac{1}{\sqrt{2}}  & 0      & 0      & \dots              & \dots               & \dots              & \dots               & \dots              & \dots               & \dots  & \dots  & 0                  & 0                   \\
        \frac{1}{\sqrt{2}} & -\frac{1}{\sqrt{2}} & 0      & 0      & \dots              & \dots               & \dots              & \dots               & \dots              & \dots               & \dots  & \dots  & 0                  & 0                   \\ \hline
        0                  & 0                   & \ddots &        & \ddots             &                     & \ddots             &                     & \ddots             &                     & \ddots &        & \vdots             & \vdots              \\
        0                  & 0                   &        & \ddots &                    & \ddots              &                    & \ddots              &                    & \ddots              &        & \ddots & \vdots             & \vdots              \\ \hline
        \vdots             & \vdots              & \ddots &        & \frac{1}{\sqrt{2}} & \frac{1}{\sqrt{2}}  & 0                  & 0                   & \ddots             &                     & \ddots &        & \vdots             & \vdots              \\
        \vdots             & \vdots              &        & \ddots & \frac{1}{\sqrt{2}} & -\frac{1}{\sqrt{2}} & 0                  & 0                   &                    & \ddots              &        & \ddots & \vdots             & \vdots              \\ \hline
        \vdots             & \vdots              & \ddots &        & 0                  & 0                   & \frac{1}{\sqrt{2}} & \frac{1}{\sqrt{2}}  & 0                  & 0                   & \ddots &        & \vdots             & \vdots              \\
        \vdots             & \vdots              &        & \ddots & 0                  & 0                   & \frac{1}{\sqrt{2}} & -\frac{1}{\sqrt{2}} & 0                  & 0                   &        & \ddots & \vdots             & \vdots              \\ \hline
        \vdots             & \vdots              & \ddots &        & \ddots             &                     & 0                  & 0                   & \frac{1}{\sqrt{2}} & \frac{1}{\sqrt{2}}  & \ddots &        & \vdots             & \vdots              \\
        \vdots             & \vdots              &        & \ddots &                    & \ddots              & 0                  & 0                   & \frac{1}{\sqrt{2}} & -\frac{1}{\sqrt{2}} &        & \ddots & \vdots             & \vdots              \\ \hline
        \vdots             & \vdots              & \ddots &        & \ddots             &                     & \ddots             &                     & \ddots             &                     & \ddots &        & 0                  & 0                   \\
        \vdots             & \vdots              &        & \ddots &                    & \ddots              &                    & \ddots              &                    & \ddots              &        & \ddots & 0                  & 0                   \\ \hline
        0                  & 0                   & \dots  & \dots  & \dots              & \dots               & \dots              & \dots               & \dots              & \dots               & 0      & 0      & \frac{1}{\sqrt{2}} & \frac{1}{\sqrt{2}}  \\
        0                  & 0                   & \dots  & \dots  & \dots              & \dots               & \dots              & \dots               & \dots              & \dots               & 0      & 0      & \frac{1}{\sqrt{2}} & -\frac{1}{\sqrt{2}} \\
      \end{array}
    \right)
  \]
\end{center}

A második fázisban a gráfon lépünk a következő csúcsba. Ahhoz, hogy balra lépjünk
a pozíció állapotvektorát egy olyan $S_{-}$ mátrixszal kell megszorozni, ami
az átló feletti 1-eseket tartalmaz:

\begin{center}
  \[ S_{-} =
    \left(
    \begin{array}{ccccc}
        0      & 1      & 0      & \cdots & 0      \\
               & \ddots & \ddots & \ddots & \vdots \\
        \vdots &        & \ddots & \ddots & 0      \\
               &        &        & \ddots & 1      \\
        0      &        & \cdots &        & 0
      \end{array}
    \right)
  \]
\end{center}

Ahhoz pedig, hogy jobbra lépjünk, egy olyan $S_{+}$ mátrixszal kell szorozni,
ami az átló alatti 1-eseket tartalmaz:

\begin{center}
  \[ S_{+} =
    \left(
    \begin{array}{ccccc}

        0      &        & \cdots &        & 0      \\
        1      & \ddots &        &        &        \\
        0      & \ddots & \ddots &        & \vdots \\
        \vdots & \ddots & \ddots & \ddots &        \\
        0      & \cdots & 0      & 1      & 0
      \end{array}
    \right)
  \]
\end{center}

Az érme $\ket{0}$ értékére lépünk balra, az $\ket{1}$ értékére pedig jobbra.
Tehát

\begin{center}
  \[ \hat{S} = S_{-}\ket{0}\bra{0} + S_{+}\ket{1}\bra{1}
  \]
\end{center}

Ami kifejtve,

\begin{center}
  \[ \hat{S} =
    \left(
    \begin{array}{cc|cc|cc|cc|cc|cc|cc}
        0                    & 0                    & \cellcolor{green!10} 1 & \cellcolor{green!10} 0 & 0                    & 0                    & \dots                  & \dots                  & \dots                  & \dots                  & \dots                & \dots                & 0                      & 0                      \\
        0                    & 0                    & \cellcolor{green!10} 0 & \cellcolor{green!10} 0 & 0                    & 0                    & \dots                  & \dots                  & \dots                  & \dots                  & \dots                & \dots                & 0                      & 0                      \\ \hline
        \cellcolor{red!10} 0 & \cellcolor{red!10} 0 & \ddots                 &                        & \ddots               &                      & \ddots                 &                        & \ddots                 &                        & \ddots               &                      & \vdots                 & \vdots                 \\
        \cellcolor{red!10} 0 & \cellcolor{red!10} 1 &                        & \ddots                 &                      & \ddots               &                        & \ddots                 &                        & \ddots                 &                      & \ddots               & \vdots                 & \vdots                 \\ \hline
        0                    & 0                    & \ddots                 &                        & 0                    & 0                    & \cellcolor{green!10} 1 & \cellcolor{green!10} 0 & \ddots                 &                        & \ddots               &                      & \vdots                 & \vdots                 \\
        0                    & 0                    &                        & \ddots                 & 0                    & 0                    & \cellcolor{green!10} 0 & \cellcolor{green!10} 0 &                        & \ddots                 &                      & \ddots               & \vdots                 & \vdots                 \\ \hline
        \vdots               & \vdots               & \ddots                 &                        & \cellcolor{red!10} 0 & \cellcolor{red!10} 0 & 0                      & 0                      & \cellcolor{green!10} 1 & \cellcolor{green!10} 0 & \ddots               &                      & \vdots                 & \vdots                 \\
        \vdots               & \vdots               &                        & \ddots                 & \cellcolor{red!10} 0 & \cellcolor{red!10} 1 & 0                      & 0                      & \cellcolor{green!10} 0 & \cellcolor{green!10} 0 &                      & \ddots               & \vdots                 & \vdots                 \\ \hline
        \vdots               & \vdots               & \ddots                 &                        & \ddots               &                      & \cellcolor{red!10} 0   & \cellcolor{red!10} 0   & 0                      & 0                      & \ddots               &                      & 0                      & 0                      \\
        \vdots               & \vdots               &                        & \ddots                 &                      & \ddots               & \cellcolor{red!10} 0   & \cellcolor{red!10} 1   & 0                      & 0                      &                      & \ddots               & 0                      & 0                      \\ \hline
        \vdots               & \vdots               & \ddots                 &                        & \ddots               &                      & \ddots                 &                        & \ddots                 &                        & \ddots               &                      & \cellcolor{green!10} 1 & \cellcolor{green!10} 0 \\
        \vdots               & \vdots               &                        & \ddots                 &                      & \ddots               &                        & \ddots                 &                        & \ddots                 &                      & \ddots               & \cellcolor{green!10} 0 & \cellcolor{green!10} 0 \\ \hline
        0                    & 0                    & \dots                  & \dots                  & \dots                & \dots                & \dots                  & \dots                  & 0                      & 0                      & \cellcolor{red!10} 0 & \cellcolor{red!10} 0 & 0                      & 0                      \\
        0                    & 0                    & \dots                  & \dots                  & \dots                & \dots                & \dots                  & \dots                  & 0                      & 0                      & \cellcolor{red!10} 0 & \cellcolor{red!10} 1 & 0                      & 0                      \\
      \end{array}
    \right)
  \]
\end{center}

A két fázist összevonva egy lépéshez a következő mátrix tartozik:

\begin{center}
  \[ \hat{U} = \hat{S} \cdot \hat{C} =
    (S_{-}\ket{0}\bra{0} + S_{+}\ket{1}\bra{1})\cdot\hat{C}
  \]
\end{center}

Ami kifejtve,

\begin{center}
  \[ \hat{U} =
    \left(
    \begin{array}{cc|cc|cc|cc|cc|cc|cc}
        0                    & 0                                      & \cellcolor{green!10} \frac{1}{\sqrt{2}} & \cellcolor{green!10} 0 & 0                    & 0                                      & \dots                                   & \dots                                  & \dots                                   & \dots                  & \dots                & \dots                                  & 0                                       & 0                      \\
        0                    & 0                                      & \cellcolor{green!10} \frac{1}{\sqrt{2}} & \cellcolor{green!10} 0 & 0                    & 0                                      & \dots                                   & \dots                                  & \dots                                   & \dots                  & \dots                & \dots                                  & 0                                       & 0                      \\ \hline
        \cellcolor{red!10} 0 & \cellcolor{red!10} \frac{1}{\sqrt{2}}  & \ddots                                  &                        & \ddots               &                                        & \ddots                                  &                                        & \ddots                                  &                        & \ddots               &                                        & \vdots                                  & \vdots                 \\
        \cellcolor{red!10} 0 & \cellcolor{red!10} -\frac{1}{\sqrt{2}} &                                         & \ddots                 &                      & \ddots                                 &                                         & \ddots                                 &                                         & \ddots                 &                      & \ddots                                 & \vdots                                  & \vdots                 \\ \hline
        0                    & 0                                      & \ddots                                  &                        & 0                    & 0                                      & \cellcolor{green!10} \frac{1}{\sqrt{2}} & \cellcolor{green!10} 0                 & \ddots                                  &                        & \ddots               &                                        & \vdots                                  & \vdots                 \\
        0                    & 0                                      &                                         & \ddots                 & 0                    & 0                                      & \cellcolor{green!10} \frac{1}{\sqrt{2}} & \cellcolor{green!10} 0                 &                                         & \ddots                 &                      & \ddots                                 & \vdots                                  & \vdots                 \\ \hline
        \vdots               & \vdots                                 & \ddots                                  &                        & \cellcolor{red!10} 0 & \cellcolor{red!10} \frac{1}{\sqrt{2}}  & 0                                       & 0                                      & \cellcolor{green!10} \frac{1}{\sqrt{2}} & \cellcolor{green!10} 0 & \ddots               &                                        & \vdots                                  & \vdots                 \\
        \vdots               & \vdots                                 &                                         & \ddots                 & \cellcolor{red!10} 0 & \cellcolor{red!10} -\frac{1}{\sqrt{2}} & 0                                       & 0                                      & \cellcolor{green!10} \frac{1}{\sqrt{2}} & \cellcolor{green!10} 0 &                      & \ddots                                 & \vdots                                  & \vdots                 \\ \hline
        \vdots               & \vdots                                 & \ddots                                  &                        & \ddots               &                                        & \cellcolor{red!10} 0                    & \cellcolor{red!10} \frac{1}{\sqrt{2}}  & 0                                       & 0                      & \ddots               &                                        & 0                                       & 0                      \\
        \vdots               & \vdots                                 &                                         & \ddots                 &                      & \ddots                                 & \cellcolor{red!10} 0                    & \cellcolor{red!10} -\frac{1}{\sqrt{2}} & 0                                       & 0                      &                      & \ddots                                 & 0                                       & 0                      \\ \hline
        \vdots               & \vdots                                 & \ddots                                  &                        & \ddots               &                                        & \ddots                                  &                                        & \ddots                                  &                        & \ddots               &                                        & \cellcolor{green!10} \frac{1}{\sqrt{2}} & \cellcolor{green!10} 0 \\
        \vdots               & \vdots                                 &                                         & \ddots                 &                      & \ddots                                 &                                         & \ddots                                 &                                         & \ddots                 &                      & \ddots                                 & \cellcolor{green!10} \frac{1}{\sqrt{2}} & \cellcolor{green!10} 0 \\ \hline
        0                    & 0                                      & \dots                                   & \dots                  & \dots                & \dots                                  & \dots                                   & \dots                                  & 0                                       & 0                      & \cellcolor{red!10} 0 & \cellcolor{red!10} \frac{1}{\sqrt{2}}  & 0                                       & 0                      \\
        0                    & 0                                      & \dots                                   & \dots                  & \dots                & \dots                                  & \dots                                   & \dots                                  & 0                                       & 0                      & \cellcolor{red!10} 0 & \cellcolor{red!10} -\frac{1}{\sqrt{2}} & 0                                       & 0                      \\
      \end{array}
    \right)
  \]
\end{center}

Az $i.$ lépés után tehát az aktuális $\Psi$ állapotvektor: $\ket{\Psi_i} =
  {\hat{U}}^i\cdot{}\ket{\Psi_0}$.

\paragraph{Mérés}

Annak a mérése, hogy a $i.$ koordinátán mekkora valószínűséggel van a bolyongó
a $j.$ lépésben.

Legyen $e_k$ egy $(2k+1)$ méretű csupa-0 vektor, kivéve az $i.$ koordinátát,
ahol 1 az értéke. Ekkor az állapotvektorból az i. koordinátához tartozó blokkot
az $\ket{e_k}\bra{e_k} \otimes \begin{pmatrix} 1 & 0 \\ 0 & 1 \end{pmatrix}$
mátrixszal történő beszorzással lehet kivágni:

\begin{center} \[ \ket{\phi_{i,j}} =
    (\ket{e_k}\bra{e_k} \otimes
    \begin{pmatrix} 1 & 0 \\ 0 & 1 \end{pmatrix})
    \cdot{} \ket{\Psi_j} \]
\end{center}

A blokkhoz tartozó valószínűség értéke az önmagával vett
(komplex) skalárszorzatának az eredménye lesz, azaz $\braket{\phi_{i,j}}$, ami valós.

Az ebben a fejezetben bemutatott algoritmust implementáltam a félév során. A 6.
fejezetben részletesen ismertetem az ezzel kapcsolatos szimulációs
eredményeket.