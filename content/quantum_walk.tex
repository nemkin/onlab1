\chapter{Kvantuminformatika}

Mielőtt a kvantumalgoritmusokkal elkezdhetünk foglalkozni, először meg kell ismernünk
a kvantuminformatika eszköztárát. A klasszikus számítógépek esetében az információtárolás
alapegysége a bit, melynek értéke lehet 0 vagy 1. Kvantumszámítógépek esetében az alapegység a kvantum bit, vagy röviden qubit. Kvantum bitek esetében a fizikai
tulajdonságaiknak köszönhetően az állapottér a komplex számokon értelmezett.
A legkisebb, kvantum számításra használható állapottér 2 dimenziós, ezt nevezzük qubitnek.

\definition{\textbf{Qubit}} A 2 dimenziós Hilbert-tér ($H_2$) egy vektorát qubitnek nevezzük. A tér bázisvektorai a $\ket{0}$ és az $\ket{1}$ ket vektorok.

Egy általános qubit tehát $c_0\ket{0}$.
\definition{\textbf{Koordináta reprezentáció}} Egy általános qubit




\definition{Kvantum regiszter}


\chapter{Kvantum séta}

Kvantum bolyongások tanulmányozása esetén először célszerű az egyenesen
vett bolyongást tanulmányozni.

