\chapter{Bevezetés}

A 20. századi fizika áttöréseivel, a relativitáselmélet és a kvantummechanika
megjelenésével gyökeresen megváltozott az emberiség addigi világnézete. Egy új
korszak kezdődött, melyet a modern fizika születéseként tartanak számon.
Ugyanebben az időszakban jelentek meg az első elektronikus számítógépek,
melynek nyomán létrejött a ,,Computer Science'', mint tudományág.

Richard P. Feynman 1982-es cikkében fejtette ki, hogy a klasszikus
számítógépekkel csak exponenciális időben lehet kvantumjelenségeket szimulálni,
ez azonban túlságosan lassú a szükséges kísérletek elvégzéséhez. Feltevése
szerint, ha létezne egy kvantumjelenségek alapján működő számítógép, akkor
azzal hatékonyan lehetne a szimulációkat futtatni. Itt jelent meg először a
kvantumszámítógép gondolata. Benioff, Deutsch, majd Bernstein és Vazirani
munkássága nyomán a 80-as évek végére megszületett a kvantum számítási modell,
a kvantum Turing-gép.

Shor 1990-es kvantum prímfaktorizációs algoritmusa miatt figyeltek fel először
a kvantumszámítógépek adta lehetőségekre. Ha valaki tudna a gyakorlatban működő
ilyen gépet készíteni, akkor az algoritmus alapján gyorsan fel tudná törni a ma
széles körben elterjedt RSA titkosítást.

A 21. században több nagyméretű vállalat is elkezdett kvantumszámítógép
építésével foglalkozni. 2019-ben a Google 54 kvantumbites Sycamore nevű
processzora sikeresen megverte a mai leggyorsabb szuperszámítógépet egy konkrét
számítás elvégzésében és ezzel elérte a kvantumfölényt.

Manapság egyre forróbb témává válik a kvantum és Magyarországon is egyre több
támogatás jut az ezzel kapcsolatos kutatásokra. A BME-n a Kvantuminformatikai
Nemzeti Labor tevékenykedik többek között kvantumtitkosításon alapuló internet
kifejlesztésével és sok tanszék, köztük a SZIT is bekapcsolódott a projektbe.

Ezen dolgozat a kvantumalgoritmusokon belül a kvantumbolyongásokkal
foglalkozik. A kvantumbolyongáson alapuló algoritmusokat azért érdemes kutatni,
mert több ismert kvantumalgoritmusnak az alapját képezik. A félév során
megismerkedtem a kvantuminformatika alapjaival, a gráfbolyongások klasszikus
illetve kvantumos változatával, valamint elkészítettem egy Pythonos
keretrendszert melyben könnyen lehet gráfbolyongásokkal kapcsolatos
kísérleteket, szimulációkat végezni.

A dokumentum hátralévő fejezeteiben bemutatom a kvantuminformatika alapjait,
ismertetem a klasszikus gráfbolyongást, továbbá a kvantumbolyongás egy
speciális esetét, bemutatom az elkészített Pythonos keretrendszert, majd
ismertetem a félév során kapott szimulációs eredményeket, végül a
továbbfejlesztési lehetőségekről beszélek.