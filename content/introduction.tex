\chapter{Bevezetés}

A 20. századi fizika hatalmas változásokat hozott. A relativitáselmélet mellett
megjelent a kvantummechanika, mely teljesen megváltoztatta a világnézetünket.
Ugyanebben az időszakban kezdődött el a számítástechnika hajnala is. Megjelentek
az első számítógépek és elkezdték megírni az első programokat, algoritmusokat.

Richard P. Feynman 1982-es cikkében fejtette ki, hogy a klasszikus számítógépekkel
sajnos csak exponenciális időben lehet kvantumjelenségeket szimulálni, ez azonban
túlságosan lassú a kísérletek elvégzéséhez. Ha viszont lenne egy kvantumjelenségek
alapján működő számítógépünk, akkor azzal hatékonyan lehetne szimulációkat végezni,
a fizikai kutatások elvégzéséhez. Így született meg a kvantumszámítógép gondolata.

Benioff, Deutsch, majd Bernstein és Vazirani munkássága nyomán megszületett a
kvantum számítási modell, a kvantum Turing-gép a 80-as évek végére. Ettől kezdve
az a kérdés foglalkoztatja a kvantumalgoritmusok kutatóit, hogy vajon vannak-e
használható kvantumalgoritmusok, illetve vannak-e olyanok amik jobbak mint a
klasszikus párjaik.

Shor 1990-es kvantumon alapuló prímfaktorizációs algoritmusa már használható
algoritmus lett és az RSA alapú kódolás feltörésével fenyeget, ami komoly
veszélyt, illetve komoly előnyt is jelent azoknak akiknek van kvantumszámítógépük.
Erre már felfigyeltek a nagyhatalmak, multinacionális cégek és szép lassan elkezdtek
a kvantumszámítógépek kifejlesztésével foglalkozni.

2019-ben a Google quantum supremacy bizonyítéka mutatott egy olyan kvantumalgoritmust
amely bár nem túl hasznos, de a mai legjobb szuperszámítógépeket is megverte a
Sicamore processzoruk teljesítménye.

Manapság egyre forróbb témává válik a kvantum és Magyarországon is egyre több
támogatás jut kvantummal kapcsolatos kutatásokra. A BME-n a Kvantuminformatikai
Nemzeti Labor tevékenykedik kvantumtitkosításon alapuló internet kifejlesztésével
és több tanszék, köztük a SZIT is bekapcsolódott a projektbe.

Ezen dolgozat a kvantumalgoritmusokon belül a kvantumbolyongásokkal foglalkozik.
A kvantumbolyongáson alapuló algoritmusokat azért érdemes kutatni, mert több
ismert kvantumalgoritmusnak az alapját képezik. A félév során megismerkedtem a
kvantuminformatika alapjaival, a gráfbolyongások klasszikus illetve kvantumos
változatával, valamint elkészítettem egy Pythonos keretrendszert melyben könnyen lehet
gráfbolyongásokkal kapcsolatos kísérleteket, szimulációkat végezni.

A dokumentum hátralévő fejezeteiben bemutatom a kvantuminformatika alapjait,
ismertetem a klasszikus gráfbolyongást, továbbá a kvantumbolyongás egy speciális
esetét, bemutatom az elkészített Pythonos keretrendszert, majd ismertetem
a félév során kapott szimulációs eredményeket, végül a továbbfejlesztési
lehetőségekről beszélek.