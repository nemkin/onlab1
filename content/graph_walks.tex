\chapter{Klasszikus gráfbolyongások}

A klasszikus gráfbolyongások tulajdonképpen Markov-folyamatoknak felelnek meg.

A Markov-folyamat valószínűségi változóknak egy sorozata ($X_1, X_2, ..., X_n, ...$),
melyek ugyanabból a diszkrét eloszlásból származnak és melyekre teljesül a
Markov-tulajdonság.

\paragraph{Markov-tulajdonság} A valószínűségi változók sorozatában
az $i+1.$ változó értéke csak az $i.$ változó értékétől függ, azaz
$P(X_{i+1} = x_{i+1} | X_i = x_i, ..., X_1 = x_1) = P(X_{i+1} = x_{i+1} | X_i = x_i)$.

\paragraph{Homogén Markov-lánc} Olyan Markov-lánc melyben az átmeneti
valószínűség nem függ az időtől, azaz $P(X_{n+1} = j | X_n = i) = p_{i,j}$.

\paragraph{Átmeneti valószínűségi mátrix}
A Markov-lánc átmeneti valószínűségi mátrixa ezen $p_{i,j}$ elemekből áll.

\paragraph{Stacionárius eloszlás} Az $X_i$ változó eloszlása ha i tart
a végtelenbe?

Egy ilyen Markov folymat megfeleltethető egy irányított súlyozott gráfnak, a következő megkötésekkel:
\begin{itemize}
  \item A gráf csúcsai megfelelnek a valószínűségi változók (közös) értékkészletének az elemeinek.
  \item Az $i.$ csúcsból a $j.$ csúcsba mutató él súlya $p_{i,j}$.
  \item Amennyiben $p_{i,j}=0$, úgy nem mutat él az $i.$ csúcsból a $j.$ csúcsba.
\end{itemize}

Ezek alapján a Markov-lánc átmeneti valószínűségi mátrixa a fent definiált gráf
szomszédossági mátrixával egyezik.

A gráfon klasszikus értelemben vett bolyongást a következő módon végzünk:
\begin{itemize}
  \item Kiindulunk egy előre megadott csúcsból.
  \item A csúcsból kifele mutató élek valószínűségeivel súlyozva véletlenszerűen
        választunk egy következő csúcsot.
  \item Meghatározott lépésig ismételjük az előző pontot.
\end{itemize}

Egy gráfon egyszerre több ilyen bolyongást is végezhetünk, párhuzamosan. A
$k.$ lépésben hozzárendelhető a gráf csúcsaihoz, hogy aktuálisan hány bolyongó
helyezkedik el bennük. Megfelelően sok bolyongóval így közelíthető az Markov-lánc
$k.$ változójának az eloszlása.

Bolyongásokkal kapcsolatban többféle tulajdonságot szoktak mérni adott gráfon.

\paragraph{Keveredési idő} A Markov lánc azon valószínűségi változójának
az indexe, melynek az eloszlása a stacionárius eloszlás $\epsilon$ sugarú környezetében
van.

\paragraph{Hitting time} Az $i \rightarrow j$ csúcsok
közötti elérési idő a legkisebb szám, ahanyadik hatványra emelve az
átmeneti valószínűségi mátrixot az $i.$ sor $j.$ cellájában nem 0 az érték?

Nagyon sok gyakorlati probléma kapcsolódik a Markov-folyamatokhoz:
közgazdaságtan, játékelmélet, statisztikus fizika. A klasszikus gráfbolyongási
algoritmusok ezek mellett jól használhatóak gráfelméleti problémák közelítő
megoldására. Nagy méretű gráfok esetén egy akár a csúcsszám méretében lineáris
algoritmus lépésszáma is túlságosan nagy lehet. Például az internetes
keresőalgoritmusok (PageRank) is ilyenek.

