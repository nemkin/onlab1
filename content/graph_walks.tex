\chapter{Klasszikus gráfbolyongások}

A klasszikus gráfbolyongások tulajdonképpen Markov-folyamatoknak felelnek meg.

A Markov-folyamat valószínűségi változóknak egy sorozata ($X_1, X_2, ..., X_n,
  ...$), melyek ugyanabból a diszkrét eloszlásból származnak és melyekre teljesül
a Markov-tulajdonság.

\paragraph{Markov-tulajdonság} A valószínűségi változók sorozatában az $(i+1).$
változó értéke csak az $i.$ változó értékétől függ, azaz $P(X_{i+1} = x_{i+1} |
  X_i = x_i, ..., X_1 = x_1) = P(X_{i+1} = x_{i+1} | X_i = x_i)$.

\paragraph{Homogén Markov-lánc} Olyan Markov-lánc melyben az átmeneti
valószínűség nem függ az időtől, azaz $P(X_{n+1} = i | X_n = j) = p_{i,j}$.

\paragraph{Átmeneti valószínűségi mátrix} A Markov-lánc átmeneti valószínűségi
mátrixa ezen $p_{i,j}$ elemekből áll.

\paragraph{Stacionárius eloszlás} Az $X_i$ változó eloszlása ha $i$ tart a
végtelenbe.

\paragraph{}

Egy ilyen Markov-folymat megfeleltethető egy irányított súlyozott gráfnak,
a következő megkötésekkel:

\begin{itemize}
  \item A gráf csúcsai ultra thinmegfelelnek a valószínűségi változók (közös)
        értékkészletének az elemeinek.
  \item A $j.$ csúcsból au $i.$ csúcsba mutató él súlya $p_{i,j}$.
  \item Amennyiben $p_{i,j}=0$, úgy feltehető, hogy nem mutat él a $j.$
        csúcsból az $i.$ csúcsba.
\end{itemize}

\paragraph{}

Ezek alapján a Markov-lánc átmeneti valószínűségi mátrixa a fent definiált gráf
szomszédossági mátrixával egyezik. A gráfon klasszikus értelemben vett
bolyongást a következő módon végzünk:

\begin{itemize}
  \item Kiindulunk egy előre megadott csúcsból.
  \item A csúcsból kifele mutató élek közül a valószínűségeikkel súlyozva
        véletlenszerűen választunk egyet, ennek a végpontja lesz a következő csúcs.
  \item Meghatározott lépésig ismételjük az előző pontot.
\end{itemize}

\paragraph{}

Egy gráfon egyszerre több ilyen bolyongást is végezhetünk, párhuzamosan. A $k.$
lépésben hozzárendelhető a gráf csúcsaihoz, hogy aktuálisan hány bolyongó
helyezkedik el bennük. Megfelelően sok bolyongóval így közelíthető az
Markov-lánc $k.$ változójának az eloszlása.

Bolyongásokkal kapcsolatban többféle tulajdonságot szoktak mérni adott gráfon.

\paragraph{Keverési idő} A Markov-lánc azon valószínűségi változójának az
indexe, melynek az eloszlása a stacionárius eloszlás $\epsilon$ sugarú
környezetében van.

\paragraph{Elérési idő} Az $i \leftarrow j$ csúcsok közötti elérési idő a
legkisebb szám, ahanyadik hatványra emelve az átmeneti valószínűségi mátrixot
az $i.$ sor $j.$ cellájában nemnulla érték szerepel.

\paragraph{}

A gyakorlatban sokféle közgazdaságtani, játékelméleti, statisztikus fizikából
ismert jelenséget modelleznek Markov-folyamatokkal. Ezek mellett gráfbolyongási
algoritmusokkal olyan problémák megoldását is fel lehet gyorsítani, a véletlen
bevezetésével, melyek alapból nem tartalmazzák azt. Például nagy méretű
gráfokban lehet ilyen módon részgráfokat keresni, de az internetes
keresőalgoritmusok (PageRank) is ezen az alapon működnek.

A kvantumbolyongásokat azért érdemes külön vizsgálni, mert esetükben a véletlen
választás teljesen másfajta módon működik, mint klasszikus esetben, erről a
következő fejezetben lesz szó.
