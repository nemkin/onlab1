\chapter{Szimulációk, eredmények}

Ábrák értelmezése:

A lenti ábrákon a matplotlibes plazma színezésnek megfelelően a hidegebb (kék) árnyalatok kisebb
értéket, a melegebb (lila, majd sárga) árnyalatok magasabb értékeket jelölnek. A fehér
szín a 0 értéket jelöli.

A gráfok és részgráfok ábrái n*n-es szomszédossági mátrixokat ábrázolnak. A fehér cellák a nem-élek,
a kék cellák az 1-es súlyú élek.

A szimulációk ábráinak az x tengelyén a gráfok csúcsai helyezkednek el kiterítve, index szerinti
növekvő sorrendben, az y tengelyen a szimuláció lépései helyezkednek el, alul a 0. lépés, felül
az utolsó lépés.

\section{Súlyzó gráf}

Például egy súlyzó gráfról a következő képek készültek:

\begin{figure}[H]
  \centering
  \begin{subfigure}{.3\linewidth}
    \centering
    \includegraphics[width=\linewidth]{./figures/sulyzo/subgraph_00.jpg}
    \caption{Bal kör}
    \label{fig:sub1}
  \end{subfigure}
  \begin{subfigure}{.3\linewidth}
    \centering
    \includegraphics[width=\linewidth]{./figures/sulyzo/subgraph_02.jpg}
    \caption{Középső teljes páros gráf}
    \label{fig:sub2}
  \end{subfigure}
  \begin{subfigure}{.3\linewidth}
    \centering
    \includegraphics[width=\linewidth]{./figures/sulyzo/subgraph_01.jpg}
    \caption{Jobb kör}
    \label{fig:sub3}
  \end{subfigure}
  \caption{Súlyzó gráf részgráfjai}
  \label{fig:all}
\end{figure}

\begin{figure}[H]
  \centering
  \includegraphics[width=0.5\linewidth]{./figures/sulyzo/graph.jpg}
  \caption{Súlyzó gráf}
\end{figure}

\begin{figure}[H]
  \centering
  \begin{subfigure}{.45\linewidth}
    \centering
    \includegraphics[width=\linewidth]{./figures/sulyzo/sim00.jpg}
    \caption{1 bolyongó}
  \end{subfigure}
  \begin{subfigure}{.45\linewidth}
    \centering
    \includegraphics[width=\linewidth]{./figures/sulyzo/sim01.jpg}
    \caption{10 bolyongó}
  \end{subfigure}
\end{figure}

\begin{figure}[H]
  \centering
  \begin{subfigure}{.45\linewidth}
    \centering
    \includegraphics[width=\linewidth]{./figures/sulyzo/sim02.jpg}
    \caption{100 bolyongó}
  \end{subfigure}
  \begin{subfigure}{.45\linewidth}
    \centering
    \includegraphics[width=\linewidth]{./figures/sulyzo/sim03.jpg}
    \caption{1000 bolyongó}
  \end{subfigure}
\end{figure}

\section{Ragasztott bináris gráf}

\begin{figure}[H]
  \centering
  \begin{subfigure}{.3\linewidth}
    \centering
    \includegraphics[width=\linewidth]{./figures/ragasztott_binaris/subgraph_00.jpg}
    \caption{Bal bináris fa}
  \end{subfigure}
  \begin{subfigure}{.3\linewidth}
    \centering
    \includegraphics[width=\linewidth]{./figures/ragasztott_binaris/subgraph_02.jpg}
    \caption{Középső teljes páros gráf}
  \end{subfigure}
  \begin{subfigure}{.3\linewidth}
    \centering
    \includegraphics[width=\linewidth]{./figures/ragasztott_binaris/subgraph_01.jpg}
    \caption{Jobb bináris fa}
  \end{subfigure}
  \caption{Ragasztott bináris gráf részgráfjai}
\end{figure}

\begin{figure}[H]
  \centering
  \includegraphics[width=0.5\linewidth]{./figures/ragasztott_binaris/graph.jpg}
  \caption{Ragasztott bináris gráf}
\end{figure}

\begin{figure}[H]
  \centering
  \begin{subfigure}{.45\linewidth}
    \centering
    \includegraphics[width=\linewidth]{./figures/ragasztott_binaris/sim00.jpg}
    \caption{1 bolyongó}
  \end{subfigure}
  \begin{subfigure}{.45\linewidth}
    \centering
    \includegraphics[width=\linewidth]{./figures/ragasztott_binaris/sim01.jpg}
    \caption{10 bolyongó}
  \end{subfigure}
\end{figure}

\begin{figure}[H]
  \centering
  \begin{subfigure}{.45\linewidth}
    \centering
    \includegraphics[width=\linewidth]{./figures/ragasztott_binaris/sim02.jpg}
    \caption{100 bolyongó}
  \end{subfigure}
  \begin{subfigure}{.45\linewidth}
    \centering
    \includegraphics[width=\linewidth]{./figures/ragasztott_binaris/sim03.jpg}
    \caption{1000 bolyongó}
  \end{subfigure}
\end{figure}

\section{Bolyongás az egyenesen, klasszikus és kvantum esetben}

\begin{figure}[H]
  \centering
  \begin{subfigure}{.45\linewidth}
    \centering
    \includegraphics[width=\linewidth]{./figures/quantum/classical_simulation_short.jpg}
    \caption{Klasszikus}
  \end{subfigure}
  \begin{subfigure}{.45\linewidth}
    \centering
    \includegraphics[width=\linewidth]{./figures/quantum/quantum_simulation_short.jpg}
    \caption{Kvantum}
  \end{subfigure}
\end{figure}

Kijött a szokásos klasszikus és kvantum közötti különbség, a klasszikus az
origó körül csúcsosodik, a kvantum viszont "2 púpú teve" szerű eloszlást mutat.

\begin{figure}[H]
  \centering
  \includegraphics[width=0.5\linewidth]{./figures/quantum/teve.png}
  \caption{Eloszlások különbsége}
\end{figure}

\section{Bolyongás a szakaszon, klasszikus és kvantum esetben}

A szakasz végpontjai visszatükrözik a bolyongót.

\begin{figure}[H]
  \centering
  \begin{subfigure}{.45\linewidth}
    \centering
    \includegraphics[width=\linewidth]{./figures/quantum/classical_simulation_long.jpg}
    \caption{Klasszikus}
  \end{subfigure}
  \begin{subfigure}{.45\linewidth}
    \centering
    \includegraphics[width=\linewidth]{./figures/quantum/quantum_simulation_long.jpg}
    \caption{Kvantum}
  \end{subfigure}
\end{figure}